\chapter{Kalibrering}
\label{cha:kalibrering}

For at kunne oversætte mellem udstyrets kanalnummer og en given energi skulle der foretages en
kalibrering. Til dette formål blev der benyttet en kilde, der bestod af tre $\alpha$-kilder:
\ce{^{241}Am}, \ce{^{239}Pu} og \ce{^{244}Cm}.

Idet forstærkningen er indstillet forskelligt i de enkelte strips og sektorer, er det
nødvendigt at kalibrere dem enkeltvis. Linierne er skarpt adskilte, derfor kan kalibreringen udføres
ved at fitte lineært til deres centroid værdier.  

Det skulle dog vise sig ikke at være helt så nemt, idet detektoren havde et dødlag, der var et par
micrometer tykt, $\Delta x$. Det har den effekt, at partikler, der rammer længere ude på detektoren, vil
miste mere energi end de, der rammer de inderste strips. De relevante størrelser er den målte energi
$E_{0}$, energien før dødlaget $E$ og vinklen $\phi$ mellem partiklens hastighed og detektorens
normal. Ud fra nogle enkelte geometriske betragtninger kan et udtryk for $E$ opskrives
\fxnote{Her skal være en tegning.}

%Lad den målte energy være
% betegnet med $E_{0}$ og energien uden dødlag være $E$. Defineres så $\phi$ som vinklen mellem
% partiklens hastighed og detektorens normal, så ses nemt at
\begin{equation}
  \label{eq:deadE}
  E = E_{0} + \d{E}{x} \frac{\Delta x}{\cos \phi}  .
\end{equation}
Det sidste led er energitabet i dødlaget, som afhænger af stoppeevnen, $\d{E}{x}$, af materialet.

\section{Estimation af dødlagets tykkelse}
\label{sec:dodlag}

For at kunne bestemme tykkelsen af dødlaget er det nødvendigt at kende kanalnumrene ved forskellige
vinkler, men som sagt er forstærkningen indstillet forskelligt i ringene.

Løsningen på dette er at udvælge en radial sektor. Hver gang denne sektor bliver ramt, findes de
cirkulære strips, hvor kanalnummeret er det samme plus minus et vist offset. Offsettet korrigerer
for at forstærkningen ikke er den samme. Kanalnummeret fra de radiale sektor tilføjes så til et
histogram for den cirkulære strip, der matchede. På denne måde er det muligt, at bestemme spektret
for de enkelte cirkulære strips, hvor forstærkningen er den samme.\fxnote{Dette skal læses igennem}

I alle disse spektrer bestemmes så centroidværdien af \Pu toppen, og kanalnummeret af denne.
Kanalnumrerne blev normaliseret til kanalnummeret i strip 1 og blev plottet som funktion af
$(\cos \phi)^{-1}$. Ud over dette er \cref{eq:deadE}, for forskellige tykkelser, også plottet. Disse er
normaliseret til det teoretiske udtryk for strip 1. Stoppeevnen er taget fra \cite{Ziegler}.

Dette er plottet på \cref{fig:dead}, for de to detektorer. Ud fra de teoretiske kurver ses, at tykkelsen
af dødlaget er konsistent med hhv. 3 og \SI{4}{\um} for upstream- og downstreamdektoreren.
\fxfatal{Der mangler usikkerhedsberegninger.}

\begin{figure}[b]
  \fxfatal{Algoritmen til at bestemme centroidværdien skal optimeres. Overvej om det skal gøres på 2D
    histogrammet. }
  \centering
  \subtop[Upstream]{\includegraphics[width=0.47\columnwidth]{DeadLayerThin}}%
  \hfill
  \subtop[Downstream]{\includegraphics[width=0.47\columnwidth]{DeadLayerThick}}%
  \caption{De normaliserede energier som funktion af $1/\cos\phi$. Kurverne angiver det teoretiske
    udtryk givet i \cref{eq:deadE} og er plottet for forskellige tykkelser.}
  \label{fig:dead}
\end{figure}


\section{Kalibreringsalgoritmen}
\label{sec:kalalgo}
\fxnote{Overvej denne overskrift}

Når tykkelsen er kendt, kan den målte energi $E_{0}$ bestemmes. Dermed kan vores detektorer
kalibreres. Fordi energitabet både afhænger af indgangsenergien og hvilken partikel, der er tale
om, så er det nødvendigt at bestemme energitabet for hver enkel hændelse.

For energier hvor stoppeevnen er stor, så vil det give anledning til en fejl, hvis energitabet anses
som konstant hele vejen igennem materialet. Det samlede tab skulle istedet udregnes som et
integral. Istedet benyttes projected range, som er middel rækkevidden af en partikel i et givent
materiale, og ækvivalent til \cref{eq:deadE} kan den samlede rækkevidde skrives som
\begin{equation}
  \label{eq:deadR}
  R(E) = R(E_{0}) + \frac{\Delta x}{\cos \phi} .
\end{equation}

For at bestemme energien af en given hændelse, så bestemmes først den samlede rækkevidde, hvor
$R(E_{0})$ bestemmes ved tabelopslag. Dernæst regnes baglæns hvormed energien bestemmes ud fra
rækkevidden. 











