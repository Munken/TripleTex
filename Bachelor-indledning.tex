\chapter{Indledning}
\label{cha:indledning}
\vspace{-0.5\onelineskip}

På baggrund af en stribe artikler udgivet af Cockcroft og Walton fra 1930 og fremefter \cite{Walton}
beskrev Oliphant og Rutherford i deres artikel, "Transmutations of Elements by Protons"{} fra 1933
\cite{Rutherford}, deres udkast til en forbedret partikelaccelerator. Disse acceleratorer muliggjorde
udforskning af nye områder inden for kernefysikken, da det var muligt at opnå intensiteter svarende
til 180\gram rent radium, hvilket tidligere havde været en af de primære kilder til energirige
partikler.

I denne artikel studerer Oliphant og Rutherford bl.a. spredning af protoner på \Bor og ud fra en
simpel model viser de, at de eksperimentelle data er i overensstemmelse med, at \Carb henfalder til
tre $\alpha$-partikler. Deres eksperimentelle data er ikke følsomme over for den specifikke
henfaldsproces, hvilket de selv påpeger. 

Forståelse af denne proces er relevant, da den giver information om trippel-$\alpha$ processen
$3\alpha\rightarrow\Carb$ i stjerner. Kort sagt kan \Carb henfalde enten direkte til tre $\alpha$-partikler
eller via en tilstand i \Be. Dette kræver præcisionsmålinger, der ikke var mulige med datidens
eksperimentelle udstyr. Udvikling inden for faststoffysik gav nye detektortyper, som tillod at måle
energien med større præcision. Disse detektorer udspænder kun en lille rumvinkel, så det er
nødvendigt at benytte flere detektorer og at flytte disse rundt for at opnå stor dækning, hvilket er
en meget tidskrævende proces. Denne viden kan også benyttes til mere jordnære formål, da
$p+\Bor \rightarrow \Carb$ er en mulig kandidat til en fusionproces, der ikke frigiver neutroner
\cite{becker}.

Den opstilling, som benyttes her i projektet, er en del af LOBENA-projektet. Her benyttes i stedet
store segmenterede faststofdetektorer, hvilket muliggør detektion af energi og position samtidig med,
at en stor rumvinkel dækkes. Formålet med LOBENA er at studere kernestrukturen af \Be, \Carb og
\ce{^{16}O} ved at populere tilstande i disse grundstoffer og studere henfaldsprodukterne. Her søges
bl.a. efter en $2^{+}$ resonans i \Carb, som blev forudsagt i 1956, men det er stadig ikke afgjort,
hvorvidt denne eksisterer.

Jeg blev tilknyttet projektet, da disse detektorer lige var blevet installeret og denne rapport vil
derfor først omhandle processen med at forstå detektorerne med henblik på at foretage
en kalibrering. Kalibreringen benyttes i analysen af de eksperimentelle data for \Carb til at
bestemme $\alpha$-spektret og såkaldte Dalitzplot, der grafisk illustrerer systemets vekselvirkninger. På
baggrund af disse argumenteres for henfaldsprocessen af \Carb.

Rapporten starter med en beskrivelse af den eksperimentelle opstilling og det eksperimentelle og
analytiske arbejde. Dernæst følger et kapitel, hvor kalibreringen af udstyret gennemgåes. I det
følgende kapitel verificeres kalibreringen. Til sidst kommer analysen af de eksperimentelle
resultater.






