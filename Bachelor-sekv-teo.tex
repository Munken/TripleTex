\subsection{Sekventielt henfald}
\label{sec:sekventielt-henfald}

End anden mulighed for tre partikel henfald er to på hinanden følgende to partikel henfald. Dette
kaldes for sekventielt henfald. I tilfældet med \Carb henfalder den først til en tilstand i \Be
under udsendelse af en $\alpha$-partikel. Modsat det direkte henfalder, som neglicerer vekselvirkninger
mellem henfaldprodukterne, så svarer dette til en kraftig vekselvirkning med to af de tre
$\alpha$-kerner. Beryllium kernen vil derefter henfalde til to sekundære $\alpha$-partikler. 

Antallet af frihedsgrader er det samme som for det direkte henfald, hvilket virker mærkeligt, da der
blot er tale om to to partikel henfald, hvor antallet af frihedesgrader er nul. Dette skyldes, at
det ikke er muligt at befinde sig i CM for begge de to henfald. Det sekventielle henfald vil derfor
både give anledning til toppe og et kontinium. .

% Med sekventielt henfald menes henfald fra den populerede tilstand i \Carb først med et primært
% henfald \Be samt en $\alpha$-partikel, hvor beryllium kernen foretager et sekundært henfald til to
% $\alpha$-partikler. Dette vil give andledning til et anderledes spektrum end et direkte henfald fra \Carb
% til tre $\alpha$-partikler, som giver andledning til et kontinium af tilstande, der afgøres af vinklen
% mellem de enkelte $\alpha$-partikler, se \cite{Becker}.

Den sekventielle henfaldsproces er illustreret på \cref{fig:becker}b, og tages udgangs i CM for
moderkernen, så vil det sekventielle henfald af \Carb give anledning til en smal top ved energien
svarende til henfald grundtilstanden af \Be.
\begin{equation}
  \label{eq:alpha0}
  \Carb* \rightarrow \Be(0^{+}) + \alpha_{0} \rightarrow 2\alpha_{2} + \alpha_{0}
\end{equation}
På trods af at beryllium-8 er ustabil er grundtilstanden stadig ret smal med en bredde på kun
\SI{6.8}{\eV}. Derfor vil bredden af toppen primært skyldes detektorens opløsningsevne. 

Ved de protonenergier, der arbejdes med, så er der endvidere mulighed for at henfalde til den første
exiterede tilstand i \Be, der ligger \SI{2.95}{\MeV} over grundtilstanden.
\begin{equation}
  \label{eq:alpha0}
  \Carb* \rightarrow \Be*(2^{+}) + \alpha_{1} \rightarrow 2\alpha_{2} + \alpha_{1}
\end{equation}
Bredden af denne top afgøres primært af bredden af beryllium tilstanden, som er
\SI{1.5}{\MeV}. Toppen vil derfor stort set svare til en Breit-Wigner fordeling, der er væsenlig
breddere, samt ligger ved lavere energi end den for $\alpha_{0}$.

Idet begge berylliumtilstande er ustabile vil der, pga. den korte levetid, forekomme endnu et
henfald udmiddelbart efter. Dette vil give andledning de to sekundære $\alpha$-partikler hhv.
$\alpha_{21}$ og $\alpha_{22}$. Energien af disse kan bestemmes ud fra følgende kinematiske overvejelser.
