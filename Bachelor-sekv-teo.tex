\subsection{Sekventielt henfald}
\label{sec:sekventielt-henfald}

En anden mulighed for tre-partikelhenfald er to på hinanden følgende to-partikelhenfald. Dette
benævnes sekventielt henfald. I tilfældet med \Carb henfalder den først til en tilstand i \Be under
udsendelse af en $\alpha$-partikel. Modsat det direkte henfald, som negligerer vekselvirkninger mellem
henfaldsprodukterne, svarer det sekventielle henfald til kraftig vekselvirkning mellem to af de tre
$\alpha$-kerner. Berylliumkernen vil derefter henfalde til to sekundære $\alpha$-partikler.

Det effiktive antal frihedsgrader er det samme som for det direkte henfald på trods af, at
sekventielt henfald blot er to to-partikelhenfald. Dette skyldes, at massemidtpunktssystemet for de
to henfald er forskellige. Set fra CM for det ene henfald vil det sekventielle henfald give
anledning til en top og et kontinuum.

% Med sekventielt henfald menes henfald fra den populerede tilstand i \Carb først med et primært
% henfald \Be samt en $\alpha$-partikel, hvor beryllium kernen foretager et sekundært henfald til to
% $\alpha$-partikler. Dette vil give andledning til et anderledes spektrum end et direkte henfald fra \Carb
% til tre $\alpha$-partikler, som giver andledning til et kontinuum af tilstande, der afgøres af vinklen
% mellem de enkelte $\alpha$-partikler, se \cite{Becker}.

Den sekventielle henfaldsproces er illustreret på \cref{fig:becker}b og tages der udgangspunkt i CM
for moderkernen, vil det sekventielle henfald af \Carb give anledning til en smal top ved energien
svarende til henfald via grundtilstanden af \Be
\begin{equation}
  \label{eq:alpha0}
  \Carb* \rightarrow \Be + \alpha_{0} \rightarrow 2\alpha_{2} + \alpha_{0}.
\end{equation}
Selvom \Be er ustabil, er grundtilstanden stadig ret smal med en bredde på kun
\SI{6.8}{\eV} \cite{States} og derfor vil bredden af toppen primært skyldes detektorens
opløsningsevne, som er 20\keV FWHM. 

Ved de anvendte protonenergier er der endvidere mulighed for at henfalde til den første exciterede
tilstand i \Be, der ligger \SI{2.95}{\MeV} over grundtilstanden
\begin{equation}
  \label{eq:alpha0}
  \Carb* \rightarrow \Be* + \alpha_{1} \rightarrow 2\alpha_{2} + \alpha_{1}.
\end{equation}
Bredden af denne top afgøres primært af bredden af \Be tilstanden, som er \SI{1.5}{\MeV}. Toppen vil
derfor approksimativt svare til en Breit-Wigner fordeling \cite[s. 26]{Martin}, som er væsenligt
breddere end den tilsvarende for $\alpha_{0}$ samt ligger ved lavere energi.

Der ligger endnu en tilstand \SI{11.35}{\MeV} over grundtilstanden. Sandsynligheden for at populere
denne er dog forsvindende lille ved de protonenergier, der benyttes. 

Idet begge berylliumtilstande er ustabile vil der pga. den korte levetid, forekomme endnu et
henfald udmiddelbart efter. Dette vil give anledning til to sekundære $\alpha$-partikler, hhv.
$\alpha_{21}$ og $\alpha_{22}$. Energien af disse kan bestemmes ud fra følgende kinematiske overvejelser.
