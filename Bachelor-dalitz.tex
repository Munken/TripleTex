\chapter{Dalitz plots}
\label{cha:dalitz-plots}

\section{Teoretisk baggrund}
\label{sec:dalitz-teori}

Faserumsvolumen eller tilstandstætheden er som navnet antyder relateret til hvilken tilstand et
system befinder sig i. Ser bort fra den rumlige orientering og udnytter at
$T_{1} + T_{2} + T_{3} = Q$, så kan kan sluttilstanden beskrives ved kun to variable $T_{1}$ og
$T_{2}$. Med denne begrænsning, så er det muligt at vise \cite[s. 120]{Bettini}
\begin{equation}
  \label{eq:DOS}
  \rho \propto \int dT_{1}dT_{2},
\end{equation}
hvor proportionaltetsfaktoren blot er en konstant faktor. Fordelingen af den kinetiske energi
relaterer sig dermed til fordelingen af tilstandstætheden.

Dette kan visuliseres grafisk. Idet sluttilstanden består af tre ens partikler og tilordning
mht. $T_{1}$ og $T_{2}$ er tilfældig, så er systemet symmetrisk. Dette kan udnyttes, idet der, for
en ligesidet trekant, gælder den geometriske egenskab, at summen af de vinkelrette afstande fra
siderne til et givent punkt $P$ er lig med højden \cite{dalitz}. Dette kendes som Vivianis sætning.

\begin{equation}
  \label{eq:dalitz}
  x = \frac{T_{1}+2T_{2}}{Q\sqrt{3}}, \hspace{3cm} y = \frac{T_{1}}{Q} - \frac{1}{3}.
\end{equation}
\Cref{eq:dalitz} viser et sæt af koordinater, således disse afstande er proportional med de
kinetiske energier. Dermed følger at punkterne, svarende til trippel $\alpha$ henfald, vil ligge inden
for trekanten grundet energibevarelse. Endvidere kan det også vises, at på grund af impulsbevarelse,
så begrænses punkterne til den indskrevne cirkel. Dette er illustreret på
\cref{fig:dalitz-triangle}.

\begin{figure}[h]
  \centering
  \fxfatal{Tegne en dalitz trekant.}
%  \includegraphics{}
  \caption{}
  \label{fig:dalitz-triangle}
\end{figure}

\subsection{Symmetribetragtninger}
\label{sec:symbetragt}

\fxfatal{Her skal findes kilder}
Hvis der ikke er nogen symmetribegrænsninger og henfaldsprodukterne ikke interagerer med hinanden,
så vil dalitzplottet være fladt. Dette skyldes at der blot vil være tale om statistisk henfald og
alle tilstande er derfor lige sandsynlige.

Henfalder systemet istedet via en resonans for derefter at henfalde til sluttilstanden, så vil det
ses på plottet som et bånd. I forhold den sekventielle henfaldsmodel og koincidensspektrene, så
forventes et bånd svarende til energien af $\alpha_{0}$ og $\alpha_{1}$. Bredden af disse bånd vil så svarer
til bredden af beryllium tilstanden, så $\alpha_{1}$-båndet bør være væsentligt breddere. Dette er
indtegnet på \cref{fig:dalitz-triangle}. Dalitzplottet kan dermed bruges til at identificerer
midlertidige resonanser og bredden af disse, på trods af at disse ikke detekteres direkte.














