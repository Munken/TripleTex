I denne afsnit er det vist, at med passende koincidensbetingelser er det muligt at ekstrahere et
samlet energispektrum, som stemmer overens hypotesen om sekventielt henfald.

Desuden er det sandsynliggjort, at tværsnittet for henfald til den første exciterede tilstand i \Be
fra \SI{17.8}{\MeV} tilstanden i \Carb er væsentligt større end tværsnittet for henfald til
grundtilstanden.

For tilstanden ved \SI{18.2}{\MeV} i \Carb er der observeret henfald til grundtilstanden af
\Be. Dette stemmer ikke overens med en $1^{+}$ isospin 0 tilstand, som det er tabuleret i
\cite{States}. De opnåede resultater indikerer istedet, at det er en $0^{+}$ eller $2^{+}$ tilstand.

Tilsvarende analyse af tilstanden ved \SI{18.5}{\MeV} tyder på en $0^{+}$ eller $2^{+}$ tilstand,
hvilket er i strid med \cite{States}, der angiver en $3^{-}$ isospin 1 tilstand. I dette tilfælde er
resultaterne af denne analyse dog problematisk. Trippelkoincidensspektret for tilstanden indikerer,
at tværsnittet for henfald til grundtilstanden af beryllium er større end det tilsvarende for
henfald til den exciterede tilstand. På trods af uoverstemmelser med \cite{States} virker
resultaterne sandsynlige, da de modsat de opgivne værdier i \cite{States} for tværsnit og
henfaldsbredder, fremstår selvkonsistente.
%
% På trods af at dette er i strid med tabellen, virker dette pålideligt, da de opgivne værdier for
% tværsnit og henfaldsbredder i tabellen ikke er konsistente.

Endvidere kan det konkluderes, at forskellen på spektrene for dobbel- og trippelkoincidenser
skyldes, at betingelserne udvælger forskellige typer henfald og at denne selektering afhænger af den
specifikke opstilling.
