Det er vist, at med passende koincidensbetingelser, er det muligt at ekstrahere et samlet
energispektrum som stemmer overens hypotesen om sekventielt henfald.  Desuden er det sandsynliggjort
at tværsnittet for henfald til den første exciterede tilstand i beryllium er væsentligt større end
tværsnittet for henfald til grundtilstanden.

Endvidere så er der redegjort for, at forskellen på spektrene for dobbelt- og
trippelkoincidenser skyldes, at disse betingelser udvælger forskellige typer henfald og at denne
selektering afhænger af den specifikke opstilling.

Det må noteres, at spektret ikke er rent nok
til, at fordelingen af de sekundære $\alpha$-partikler kan bestemmes.
