\subsection{Koincidens}
\label{sec:koincidens}
For at grovsortere data for protonhændelser var det logiske kredsløb indstillet til \lAND således,
at der kun forekom en hændelse, når to eller flere detektorer blev ramt. Der kan dog stadigvæk
forekomme tilfældige koincidenser i data. Arbejdet bestod derfor i at bestemme de
koincidenser, som svarede til trippel-$\alpha$ henfald.

For at sortere tilfældige fluktuationer i udstyret fra, blev detektioner i forsiden af detektoren
sammenholdt med dem på bagsiden. Idet detektorerne består af et stykke silicium med kontakter på
hhv. for- og bagside, vil en reel partikel blive detekteret i begge sider. En detektion blev kun
accepteret, hvis der kunne findes en med tilsvarende energi i bagsiden, dog inden for en vis
tolerance.

\paragraph{Trippelkoincidenser}
\label{sec:trippelkoincidenser}
Hvis der i en given hændelse blev detekteret tre eller flere partikler, blev der ledt efter
trippelkoincidenser. Disse kan forekomme på to måder i detektorerne. Enten vil partiklerne have ramt
hver sin detektor eller også vil to have ramt den samme, mens den tredje rammer en anden
detektor. Hvis den samlede energi af en sådan triplet er lig med Q-værdien, inden for en vis
tolerance, er det formentlig et sæt af tre matchende $\alpha$-partikler
\begin{equation}
  \label{eq:trippelE}
  T_{1} + T_{2} + T_{3} = Q \pm \Delta E.
\end{equation}
Det kriterie kan forbedres ved at tage højde for impulsen. Idet detektorerne er positionsfølsomme, er
det muligt at udregne azimut- og polarvinklen af hver partikel. Impulsen kan så udregnes fra
energien $p \propto \sqrt{T}$. For hver triplet udregnes den samlede impuls i CM, hvor den per
konstruktion skal være nul
\begin{equation}
  \label{eq:pCMnul}
  \mvec{p_{1}} + \mvec{p_{2}} + \mvec{p_{3}} = \mvec{0}.
\end{equation}
Ud fra impulsvektorerne er det også muligt at tjekke to andre krav. Er der tale om en sand
koincidens skal summen af vinklerne vektorerne imellem være lig 360\degree. Endvidere foregår
henfaldet i ét plan, hvilket matematisk kan formuleres som
\begin{equation}
  \mvec{p_{3}} \cdot (\mvec{p_{1}} \times \mvec{p_{2}}) = \mvec{0}.
\end{equation}

\paragraph{Dobbeltkoincidenser}
\label{sec:dobbeltkoincidenser}

Hvis spektret er meget rent, er det også muligt at benytte dobbeltkoincidenser. Såfremt der kun blev
detekteret to partikler i en hændelse, er de formentlig to tredjedele af en triplet. Energien af den
sidste er derfor differencen mellem Q-værdien og den samlede energi af de detekterede partikler
\begin{equation}
  \label{eq:dobbeltE}
  T_{3} = Q - T_{1} - T_{2}.
\end{equation}

Det er en afvejning, om disse skal medtages. Antallet af detekterede koincidenser øges kraftigt, men
samtidig giver de også anledning til fejl. Nogle åbenlyse fejl kan reduceres ved at tjekke, hvorvidt
energien af den tredje partikel bliver negativ samt at $x$ og $y$ for Dalitzplottet ligger inden for
den indskrevne cirkel.

Har man udregnet impulsvektorerne for de to detekterede partikler, kan der indføres
et ekstra krav til dobbeltkoincidenserne. Den samlede impuls i CM skal være nul. Den ukendte
impuls kan findes ud fra de to detekterede partiklers impuls
\begin{equation}
  \mvec{p_{3}} = -(\mvec{p_{1}} + \mvec{p_{2}}).
\end{equation}
Den kinetiske energi kan også findes ud fra impulsens størrelse, hvilket leder til kravet om, at de to
estimater skal stemme overens
\begin{equation}
  T_{3} = \frac{\lVert\mvec{p_{1}} + \mvec{p_{2}}\rVert^{2}}{2m_{\alpha}}.
\end{equation}
