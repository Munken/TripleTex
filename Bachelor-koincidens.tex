\subsection{Koincidens}
\label{sec:koincidens}
For at grovsortere data for protonhændelser, var det logiske kredsløb indstillet til \lAND, således
at der kun forekom en hændelse, når begge de runde detektorer blevt ramt. Der kan dog stadigvæk
forekomme tilfældige koincidenser. Arbejdet bestod derfor i at bestemme de koincidenser, som svarede
til trippel-$\alpha$ henfald.

Hvis der i en given hændelse blev detekteret tre eller flere partikler, så blev der ledt efter
tripelkoincidenser. Disse kan forekomme på to måder i detektorerne. Enten vil partiklerne have ramt
hver sin detektor eller også vil to have ramt den samme, hvor den tredje så har ramt en anden
detektor. Hvis den samlede energi af en sådan triplet er lig med Q-værdien, så er det formentlig et
sæt af tre matchende $\alpha$-partikler. 

Hvis spektret er meget rent, er det også muligt at benytte dobbelkoincidenser. I det tilfælde hvor
der i en hændelse kun blev detekteret to partikler, så er de formentlig to tredjedele af en
triplet. Energien af den sidste er så differencen mellem Q-værdien og den samlede energi af de
detekterede partikler. Det er en afvejning af disse skal tages med. De øger antallet af detekterede
koincidenser kraftigt, men samtidig giver de også andledning til fejl. Nogle åbenlyse fejl, såsom at
tjekke om energien af den tredje bliver negativ, opfyldelse af impulsbevarelse kan afhjælpe nogle af
problemerne.

På trods af denne matching forekom, der stadig støj i spektret. Støjen var dog lokaliseret til ikke
kritiske energier, så det var muligt blot at skære disse væk.