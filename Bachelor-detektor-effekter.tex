\subsection{Detektoreffekter}
\label{sec:detektor-effekter}

Ud fra diskussionen i \cref{sec:sekv-kinematik}, især med \cref{eq:Ealpha2} og \cref{eq:sekv-vinkel}
in mente, ses det, at energien til rådighed for de sekundære $\alpha$-partikler i den transversale retning
afhænger af Q-værdien af det sekundære henfald.

Et henfald fra grundtilstanden af beryllium til tre $\alpha$-partikler frigør cirka 90\keV, hvorimod et
henfald fra den exciterede tilstand frigør omkring 3\MeV. Dette betyder, at den transversale
komposant af hastigheden af de sekundære $\alpha$-partikler kan være mange gange større, hvormed vinklen
mellem de to sekundære $\alpha$-partikler kan være tilsvarende større. Benytter man
\cref{eq:sekv-vinkel}, er den maksimale vinkel hhv. \SI{19}{\degree} og \SI{86}{\degree} for
henfald til grundtilstanden og den exciterede tilstand. Vinklen mellem detektorerne, set fra
foliet, er i størrelsesordnen 20\degree.

Dette betyder, at dobbeltkoincidenserne primært vil udgøres af $\alpha_{1}$ og dens sekundære partikler,
da sandsynligheden for, at en af disse rammer ved siden af, er større pga. den større vinkel mellem
de sekundære partikler. Der vil også være nogle få $\alpha_{0}$ i dobbeltkoincidenserne. Disse vil dog
være kraftigt undertrykt, da den lille vinkel gør, at begge de sekundære partikler enten vil ramme
en detektor eller flyve ved siden af.

Trippelkoincidenserne kan forekomme på to måder. Enten rammer de sekundære partikler samme
detektor, ellers rammer de hver deres. Det er klart, at på grund af åbningen imellem detektorerne,
vil det udelukkende være $\alpha_{1}$ og dens sekundære partikler, der registreres i tre
dektorer. Hvis de sekundære partikler registreres i samme detektor, kan dette enten være
resultatet af $\alpha_{0}$- eller $\alpha_{1}$-henfald, da \cref{eq:sekv-vinkel} angiver en vinkelfordeling,
der strækker sig fra 0 og op til maksimalværdien. Produceres der samme mængde $\alpha_{0}$ og
$\alpha_{1}$, så må det forventes, at $\alpha_{0}$ dominerer denne kanal, da det er mere sandsynligt, at
vinklen mellem dens sekundære partikler er lille.


% \fxfatal{Dette skal gennemtænkes.}
% Med det benyttede dektorsystem er der ikke fuld dækning i alle retninger, men fordi detektorerne er
% placeret symmetrisk, så er detektorerne mere effektive til at detektere hændelser med lille vinkel
% mellem de sekundære $\alpha$-partikler, da disse to ofte ville ramme samme detektor. Derimod er sandsynligheden
% større for at kun to $\alpha$-partikler bliver detekteret, hvis vinklen er større, da der så er mulighed
% for at en af partiklerne slet ikke detekteres.

% Hvad betyder dette for koincidensspektrene? Effekten er tydeligst ved dobbeltkoincidenserne. Her
% undertrykkes $\alpha_{0}$ kraftigst i forhold til $\alpha_{1}$. Dette skyldes, som beskrevet ovenfor, at der
% vil være forholdvis flere hændelser, der skyldes henfald til den exciterede tilstand, hvor \emph{kun} to
% partikler detekteres. Ved specifik at vælge disse hændelser undertrykkes $\alpha_{0}$-toppen.

% Hvorfor er $\alpha_{1}$ så ikke undertrykt i trippelkoincidensspektret? Her er der to effekter, som
% spiller ind. For det første er vinklen mellem detektorerne i størrelsesordnen 20\degree, hvilket
% betyder, at der er en risiko for at begge sekundære partikler for $\alpha_{0}$ henfaldet rammer ved siden
% af. Muligheden for større vinkel ved $\alpha_{1}$ henfald betyder også, at der er mulighed for, at
% $\alpha$-partiklerne rammer hver sin detektor. Hvordan disse effekter har indflydelse på spektret er svær
% at afgøre, men generelt er sandsynligheden for dektektere $\alpha_{1}$ tripler mindre end den for
% $\alpha_{0}$.