\message{ !name(Bachelor-sekv-konklusion.tex)}
\message{ !name(Bachelor-sekv-konklusion.tex) !offset(11) }
Trippelkoincidensspektret for tilstanden indikerer, at tværsnittet for henfald til
grundtilstanden af beryllium er større end det tilsvarende for henfald til den exciterede
tilstand. På trods af uoverstemmelser med \cite{States} virker resultaterne sandsynlige, da de
modsat de opgivne værdier i \cite{States} for tværsnit og henfaldsbredder, fremstår
selvkonsistente. På trods af dette, er der uregelmæssigheder i Dalitzplottet, så det er nødvendigt
med yderligere analyse før dette kan afgøres.
\message{ !name(Bachelor-sekv-konklusion.tex) !offset(-9) }
