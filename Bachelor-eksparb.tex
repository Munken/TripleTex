\chapter{Det eksperimentelle arbejde}
\label{cha:eksp}

Det eksperimentelle arbejde begyndte i januar 2013. På dette tidspunkt var kun de to S3 detektorer
monteret. Her blev foretaget målinger med protonenergier mellem \num{0.5} og \SI{2.4}{\MeV}, hvor
det logiske kredsløb var indstillet til både \lAND og \lOR. Dette gav mulighed for at undersøge både
Rutherfordspredning og $\alpha$-henfald. Detektionsraten, når det logiske kredsløb var indstillet på \lOR,
var mellem 10 og 20\kHz. Af disse blev cirka 20\% afvist, da de ankom, mens analog-til-digital
konverteringen var i gang. Med det logiske kredsløb indstillet til \lAND blev ingen hændelser afvist,
da detektionsraten kun var omkring 200\Hz. Sandsynligheden for, at $\alpha$-partikler blev afvist, fordi
ADK'en var optaget, var derfor meget mindre.

Til at foretage en kalibrering mellem energi og kanalnumre blev der benyttet tre $\alpha$-kilder. Det
viste sig dog hurtigt, at der var problemer med kalibreringen. Hvis der blev benyttet 2\MeV
protoner, forekom den tilsvarende Rutherfordtop i spektret for den inderste ring i detektor 4 ved
\SI{2.6}{\MeV}. Forklaringen på dette var, at detektorerne havde et inaktivt område, et såkaldt
dødlag, hvor partiklerne tabte energi, før de blev detekteret. Fordi kalibreringskilden var en
$\alpha$-kilde og stoppeevnen af disse er større end protoners, blev der overkompenseret for dødlaget.

På baggrund af denne observation blev der foretaget nye målinger med kalibreringskilden, så
tykkelsen af det inaktive område kunne fastslås. Selve proceduren findes i
\cref{cha:kalibrering}. Resultatet af målingerne viste, at dødlaget på den side af detektorerne, der
hidtil havde vendt ind mod foliet, var væsentligt tykkere end det på bagsiden af detektorerne. 

For at mindske usikkerheden blev detektorerne roteret og kalibreret, hvor der blev taget hensyn til
dødlaget.  I mellemtiden var de to W1 detektorer også blevet tilføjet. Kalibreringerne af alle
detektorerne bliver verificeret i \cref{cha:rutherford}. De originale data blev kasseret og nye
målinger ved 2, \num{2.37} og \SI{2.65}{\MeV} blev foretaget. Disse nye data blev alle
indsamlet med det logiske kredsløb indstillet til \lAND og over et længere tidsrum for at øge
datamængden. Den eneste ulempe var, at der ikke blev foretaget målinger over et ligeså bredt
energiinterval. Analysen af disse data findes i \cref{cha:sekventielt-henfald}.

\section{Analysearbejdet}
\label{sec:analyse}

Til analysearbejdet blev analyseværktøjet ROOT benyttet. ROOT er udviklet på CERN og er en række
udvidelser til programmeringssproget C++. De rå datafiler blev konverteret til en ROOT specifik
datastruktur, kaldet et ROOT-træ. Indlæsning af data foregår med en standardalgoritme og foregår en
hændelse ad gangen. For hver hændelse registreres for hver detektor antal detektioner i
forsiden og bagsiden af detektoren, de ramte strips samt kanalnummeret for de enkelte
detektioner. Opgaven var  at rekonstruere de fysiske begivenheder ud fra disse tal.

Først og fremmest er det nødvendigt at foretage en kalibrering for hver enkelt strip, så
kanalnumrene kan oversættes til kinetisk energi. Endvidere kan både azimut- og polarvinklen
rekonstrueres ud fra den ramte for- og bagstrip. Dette gøres ud fra både detektortypen og afstanden
til detektoren. Det er ikke muligt at bestemme afstanden til detektorerne med stor nøjagtighed, da
opstillingen er meget kompakt og detektorerne skrøbelige. Derfor var det nødvendigt at justere
parametererne for placeringen af de forskellige detektorer indtil resultaterne var selvkonsistente
samt konsistent med veletableret fysik, såsom energiens vinkelafhængighed for Rutherfordspredte
protoner. Dette er beskrevet i \cref{cha:rutherford}.

For hver hændelse blev energien og vinklerne af de enkelte partikler beregnet. Resultaterne i dette
projekt er fundet ved kombinere disse informationer som beskrevet i hhv. \cref{sec:ruther-data} og
\ref{sec:sek-data}.

Koden brugt til analysen er skrevet fra bunden i løbet af projektet.
