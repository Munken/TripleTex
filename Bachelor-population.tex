\subsection{Tilstandspopulering}
\label{sec:tilstand}

\subsubsection{Kulstof-12}
\label{sec:pop-carbon}

Som tidligere nævnt bevarer $\alpha$-henfald både spin og paritet. Dermed er det muligt at sammenholde
Dalitzplottet for henfaldsprodukterne med den passende skitse på \cref{fig:dalitz-0}. Det er dog kun
muligt, hvis den populerede \Carb tilstand er kendt. Dette er f.eks. tabuleret i f.eks. \cite{States}, hvor
tilstanden er givet ud fra både proton- og excitationsenergien. Protonenergien kan aflæses på
acceleratoren, mens excitationsenergien skal udregnes.

Til følgende diskussion kan med fordel bladres tilbage til \cref{fig:becker} på
\cpageref{fig:becker}.

Hvis både protoner og \Be er i hvile, er excitationsenergien givet ved masseforskellen
\begin{equation}
  \label{eq:massdiff}
  \Delta m = m_{\ce{Be}} + m_{p} - m_{\ce{C}} = \SI{15.96}{\MeV}. 
\end{equation}
Dette medtager ikke den kinetiske energi $T_{p}$ af protonen. Energien tilgængelig for
excitationen er den energi, som ikke er bundet i massemidtpunktsbevægelsen. Denne findes ud fra den
kinetiske energi af massemidtpunktet
\begin{equation}
  T_{CM} = \frac{1}{2}MV_{CM}^{2},
\end{equation}
hvor $M$ er den samlede masse. Hastigheden af massemidtpunktet er givet ved
$V_{CM} = \frac{\mu}{m_{\ce{Be}}} V_{p}$. Protonhastigheden kan findes ud fra dens kinetiske energi, hvorved den
kinetiske energi af massemidtpunktet kan skrives som
\begin{equation}
  \label{eq:Kcm}
  T_{CM} = \frac{m_{p}}{m_{p} + m_{\ce{Be}}} T_{p} \approx \frac{1}{12} T_{p}.
\end{equation}
Den samlede excitationsenergi er dermed
\begin{equation}
  \label{eq:excitation}
  E^{*} = \Delta m + T_{p} - T_{CM} = \Delta m + \frac{11}{12} T_{p},
\end{equation}
hvilket er i omegnen af 18-19\MeV, da der er benyttet 2-3\MeV protoner til målingerne.

\subsubsection{Beryllium-8}
\label{sec:pop-beryllium}

Grundtilstanden af \Be er en $0^{+}$ tilstand, mens den første exciterede tilstand er en $2^{+}$
tilstand. Hvilke af disse, der populeres, afhænger af kulstoftilstanden $J^{\pi}$.

Kulstofkernen henfalder til $\alpha$ og \Be, hvis samlede bølgefunktion har spin $L$.  $\alpha$-kernen er en
spin 0 boson. For at opfylde paritet- og impulsbevarelse skal følgende gælde
\begin{equation}
  \label{eq:spin}
  \pi_{\!\!\Carb} = (-1)^{L} \pi_{\!\Be}  \qquad\qquad \mvec{J_{\!\!\Carb}} = \mvec{J_{\!\Be}} + \mvec{L}.
\end{equation}

Hvis kulstof er i en naturlig paritets tilstand, $\pi = (-1)^{J}$, så er ovenstående opfyldt for
$0^{+}$ tilstanden, såfremt $L = J$. Tilsvarende er $L = J$ også en løsning for $2^{+}$. Her skal
man blot huske på, at spin er en vektor.

Hvis kulstof i stedet er i en unaturlig paritets tilstand, $\pi = (-1)^{J+1}$, så er $L = J\pm1$ en
løsning for $2^{+}$ tilstanden. Det er dog ikke muligt at opfylde både paritets- og impulsbevarelse
for henfald til $0^{+}$ tilstaden.

På baggrund af dette forventes både $\alpha_{0}$- og $\alpha_{1}$-bånd, hvis kulstof har naturlig paritet, mens
der kun forventes $\alpha_{1}$-bånd ved unaturlig paritet.

Endvidere skal henfaldet også opfylde bevarelse af isospin. Da $\alpha$-partiklen er en isospin 0
partikel, kan \Carb kun henfalde til tre $\alpha$-partikler, hvis den selv er i en isospin 0
tilstand. Forekommer henfaldet på trods af dette, må tilstanden have haft en isospin 0 komponent.






