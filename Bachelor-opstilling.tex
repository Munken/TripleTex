\chapter{Opstilling}
\label{cha:opstilling}

Et beam af protoner accelereres op til den ønskede energi med en 5\MeV Van de Graaf
accelerator. Dette beam afbøjes med en eletromagnet og sendes ind i beamline, hvor det først
passerer gennem et hul i midten af den ene detektor, hvorefter en del af det vil kolliderer med et
\ce{^{11}B}-target på carbon backing.
\fxfatal{Hvad er tykkelsen af foliet og backing?}
Det resterende vil passere videre gennem beamline, hvor det
igen vil passere gennem en detektor, for at ende i en Faraday cup.

Detektorsystemet består af to dobbel siddet silicium strip detektorer (DSSSD), som fungerer på samme
måde, som almindelige fasstofdetektorer. Fordelen er opdelingen af forsiden og bagsiden i et antal
områder kaldet strips. Dermed er det muligt at bestemme både vinkel og energi af
partiklerne. Bagsiden er opdelt i 32 radiale slices, som benævnes sektorer. Forsiden er derimod
opdelt i en række ringe, der hver er 886\um tykke, med et 100\um isolerende område mellem hver.

Den første detektor beamet passerer igennem kaldes upstream. Denne udspænder vinklerne fra
$141\degree$ til $165\degree$ målt fra beamets retning. Den anden benævnes med downstream og den
udspænder fra $15\degree$ til $40\degree$.


\begin{figure}[h]
  \centering
  \fxfatal{Skematisk tegning af opstillingen mangler.}
  %\includegraphics{}
  \caption{Skematisk tegning af opstillingen}
  \label{fig:opstilling}
\end{figure}
