\begingroup
\let\clearpage\relax
\chapter{Abstract}
\label{cha:abstract}
\vspace{-0.5\onelineskip}
I the course of history the $p + \Bor \rightarrow \Carb*$ process have been studied in great detail
because it unveils information about the triple-$\alpha$ process.

In this project the triple-$\alpha$ decay has been studied with four segmented solidstate detectors. This
allows for simultaneous detection of the energy and position of the daugther particles. This type of
detector is rather new, so an important part of the project was to understand the detectors. Methods to
overcome these difficulties in presented.

It was found that \Carb decays to three $\alpha$-particles via a state in \Be. It has
been determined that the state at \SI{17.8}{\MeV} in \Carb is either a $0^{+}$ or $2^{+}$
state. From the spectrum it was possible to deduct that decay via the excited state in \Be is more
probable that decay via the groundstate. These observations is cosistent with the literature \cite{States}.

Furthermore the results suggests that at \num{18.2} and \SI{18.5}{\MeV} \Carb is in a $0^{+}$ or
$2^{+}$ state which contradict the literature. The conclusion is that there is no $1^{+}$ state at
\SI{18.2}{\MeV}, but the results was not conclusive for the \SI{18.5}{\MeV} state.

These results shows that segmented solidstate detectors provides an improved tool for observing
nuclearprocesses with multiple decayfragments. 

\vspace{1.5\onelineskip}
\chapter{Resume}
\label{cha:resume}
\vspace{-0.5\onelineskip}

Historisk set er $p + \Bor \rightarrow \Carb*$ processen blevet studeret i stor detalje, da den
giver mulighed for at forstå trippel-$\alpha$ processen.

Processen er i denne rapport undersøgt ved at anvende fire segmenterede faststofdetektorer, som
muliggør energi- og positionsfølsom dektektion. Denne type detektorer er forholdsvis ny, hvilket gav
anledning til en række vanskeligheder i starten af projektet. Der vil blive redegjort for
løsningen på disse problemer. 

% Der vil blive redegjort for disse, samt metoder til at løse disse
% vaskeligheder. \fxfatal{Indforstået - flere ord}

Resultatet af undersøgelserne er, at \Carb henfalder til tre $\alpha$-partikler via en tilstand i
\Be. Det kan konkluderes, at der findes en $0^{+}$ eller $2^{+}$ tilstand ved \SI{17.8}{\MeV} i
\Carb. Endvidere blev det observeret, at tværsnittet for henfald til grundtilstanden er større end for
henfald til den exciterede tilstand. Dette stemmer overens med litteraturen \cite{States}.

Undersøgelserne indikerer dog også en $0^{+}$ eller $2^{+}$ tilstand ved hhv. \num{18.2} og
\SI{18.5}{\MeV}, hvilket er i strid med litteraturen. For tilstanden ved \SI{18.2}{\MeV} er
fortolkningen, at den rapporterede $1^{+}$ tilstand ikke findes. Resultaterne for \SI{18.5}{\MeV} er
dog ikke endegyldige.

Resultaterne viser, at segmenterede faststofdetektorer giver mulighed for at observere
nukleareprocesser med flere datterkerner med større præcision end hidtil muligt. 




\endgroup
