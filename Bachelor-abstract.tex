\begingroup
\let\clearpage\relax
\vspace*{-5.5\onelineskip}
\chapter{Abstract}
\label{cha:abstract}
\vspace{-0.5\onelineskip}

In the course of the 20th century the $p + \Bor \rightarrow \Carb*$ reaction have been studied in
considerable detail because this reaction allows to study several questions, such as the decay
mechanism to the $3\alpha$ final states, and position of resonances in \Carb. This reaction is also a
candidate for fusion energy reactors, with the advantage that no neutrons are produced.

In this project this reaction has been studied with four segmented solid state detectors. This allows
for simultaneous detection of the energy and position of the daugther particles. This type of
detector is rather new, so an important part and motivation for the project was to understand the
detectors. Methods to overcome difficulties with these detectors is presented.

It was found that \Carb decays to three $\alpha$-particles via states in \Be. It has been determined that
the state at \SI{17.8}{\MeV} in \Carb is either a $0^{+}$ or $2^{+}$ state. From the spectrum it was
possible to deduce that decay via the excited state in \Be is more probable than decay via the
groundstate. These observations are consistent with the literature \cite{States}.

Furthermore the results suggest that at \num{18.2} and \SI{18.5}{\MeV} \Carb is in a $0^{+}$ or
$2^{+}$ state which contradict the literature. The conclusion is that there is no $1^{+}$ state at
\SI{18.2}{\MeV}, but the results were not conclusive for the \SI{18.5}{\MeV} state.

These results shows that segmented solid state detectors provides an improved tool for observing
nuclear processes with multiple decay fragments. 

\vspace{1\onelineskip}
\chapter{Resume}
\label{cha:resume}
\vspace{-0.5\onelineskip}

I løbet af det 20. århundrede er $p + \Bor \rightarrow \Carb*$ reaktionen blevet studeret i stor
detalje, da den giver mulighed for undersøge adskillige spørgsmål, såsom henfaldsmekanismen til $3\alpha$
sluttilstande og positionen af resonanser i \Carb. Desuden er reaktionen også en kandidat til
fremtidens fusionsreaktorer, da den ikke producerer frie neutroner. 

I denne rapport er reaktionen undersøgt ved at anvende fire segmenterede faststofdetektorer, som
muliggør energi- og positionsfølsom dektektion. Denne type detektorer er forholdsvis ny, så en
vigtig del og motivation for projekt var at forstå disse detektorer. Der vil blive redegjort for
løsningen problemer i forbindelse med detektorsystemet.

% Der vil blive redegjort for disse, samt metoder til at løse disse
% vaskeligheder. \fxfatal{Indforstået - flere ord}

Resultatet af undersøgelserne er, at \Carb henfalder til tre $\alpha$-partikler via en tilstand i
\Be. Det kan konkluderes, at der findes en $0^{+}$ eller $2^{+}$ tilstand ved \SI{17.8}{\MeV} i
\Carb. Endvidere blev det observeret, at tværsnittet for henfald til grundtilstanden er større end for
henfald til den exciterede tilstand. Dette stemmer overens med litteraturen \cite{States}.

Undersøgelserne indikerer også en $0^{+}$ eller $2^{+}$ tilstand ved hhv. \num{18.2} og
\SI{18.5}{\MeV}, hvilket er i strid med litteraturen. For tilstanden ved \SI{18.2}{\MeV} er
fortolkningen, at den rapporterede $1^{+}$ tilstand ikke findes. Resultaterne for \SI{18.5}{\MeV} er
dog ikke endegyldige.

Resultaterne viser, at segmenterede faststofdetektorer giver mulighed for at observere
nukleareprocesser med flere datterkerner med større præcision end hidtil muligt. 




\endgroup
