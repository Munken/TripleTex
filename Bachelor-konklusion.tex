\chapter{Samlet konklusion}
\label{cha:konklusion}

I dette projekt er det vist, at det er muligt at detektere henfaldsprodukterne fra $\Carb
\rightarrow 3\alpha$ processen med segmenterede faststofdetektorer, hvis der tages højde for
eventuelle dødlag, der måtte være på de enkelte detektorer. Metoder til at estimere tykkelsen af
disse er fremlagt.


Det er vist, at det er muligt at detektere både dobbelt- og trippelkoincidenser med passende
koincidensbetingelser. Ud fra disse koincidenser var det muligt at bestemme $\alpha$-spektret. Spektret
var i overstemmelse med, at \Carb henfalder til tre $\alpha$-partikler via en tilstand i \Be. Det var
endvidere muligt at bestemme spektre for henfald via de enkelte tilstande. Ud fra spektrene var det
muligt at fastslå, at der ikke findes en $1^{+}$ isospin 0 tilstand ved \SI{18.2}{\MeV} i \Carb.

Endvidere er der redegjort for teorien bag Dalitzplottet og ved analyse af sådanne er det fundet, at
der ved \num{17.8}, \num{18.2} og \SI{18.5}{\MeV} findes en $0^{+}$ eller en $2^{+}$ tilstand. For
tilstanden ved \SI{17.8}{\MeV} er dette konsistent med litteraturen, hvilket dog ikke er tilfældet
for tilstanden ved \SI{18.5}{\MeV}. Her er yderligere analyse dog nødvendig, hvilket kunne være at
bestemme vinkelfordelingen af $\alpha$-partiklerne, der også afhænger af \Carb-tilstanden. Endvidre ville
det også være interessant at foretage målinger over et stort energiinterval for at bestemme
eventuelle baggrundstilstande.

Der er i dette projekt ikke foretaget Monte Carlo simuleringer. Dette er dog nødvendigt, hvis der
ønskes en grundigere analyse, da der skal korrigeres for effekter, som skyldes den eksperimentelle
opstilling. Foretages disse korrektioner, er det muligt at bestemme absolutte tværsnit for henfald
til grundtilstanden og første exciterede tilstand, hvilket er en nødvendig viden hvis denne reaktion
til skal bruges til fusion.  

Til sidst skal nævnes, at det med den anvendte analysemetode også er muligt at studere henfald, hvor
der først udsendes en foton. Dette gøres ved at lede efter triplpelkoincidenser, der overholder
impulsbevarelse, men hvor den samlede energi ikke er lig energien frigivet ved $\Carb* \rightarrow 3\alpha$.







